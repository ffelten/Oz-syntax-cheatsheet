\documentclass[12pt]{article}
\usepackage[english]{babel}
\usepackage[utf8x]{inputenc}
\usepackage{fullpage}
\usepackage{multicol}
\usepackage{listings}
\usepackage{color}
\usepackage{longtable}
\usepackage{booktabs}
\usepackage{hyperref}
\usepackage{fancyhdr}
\usepackage{listings}
\usepackage[table]{xcolor}

\pagestyle{fancy}
\fancyhf{} %Clean header and footer 
\rhead{Last compiled: \today}
\lhead{Université Catholique de Louvain}
\rfoot{\thepage}
\lfoot{Florian Felten}

%\title{Oz Programming: Basic syntax cheatsheets}
\author{Felten Florian}

\definecolor{mygreen}{rgb}{0,0.6,0}
\definecolor{mygray}{rgb}{0.5,0.5,0.5}


\lstdefinestyle{myOz}{
	language=oz,
	backgroundcolor=\color{white},   % choose the background color
  	basicstyle=\ttfamily\small,        % size of fonts used for the code
  	breaklines=true,                 % automatic line breaking only at whitespace
  	captionpos=b,                    % sets the caption-position to bottom
  	commentstyle=\color{mygray},    % comment style
  	escapeinside={\%*}{*)},          % if you want to add LaTeX within your code
  	keywordstyle=\color{blue},       % keyword style
  	stringstyle=\color{mygreen},    % string literal style
  	showstringspaces=false
%doesn't work  	morekeywords={<, =<, >, >=, ==, \=, @, :=}
	}

\lstnewenvironment{oz}
  {\lstset{style=myOz}}
  {}



\begin{document}

\vspace*{2em}
\begin{center} % Title
	{\Large Oz Programming: Basic syntax cheatsheet}
\end{center}

This document is a non-exhaustive reminder of the syntax of the Oz programming language. It is always possible to improve it and your help is therefore welcome -- just submit an issue on the link below and we will modify the document. Source code and the latest version of the pdf can be found at the following address:
\url{https://github.com/el-nounou/Oz-syntax-cheatsheet}

\renewcommand*{\arraystretch}{1.5}
\begin{longtable}{l r}
\toprule[0.2em]
\multicolumn{1}{l}{\textbf{Keyword}} & \textbf{Meaning}\\
\midrule[0.3mm]


 %%%%% 
\multicolumn{2}{c}{\textbf{Basic statements}}\\
\midrule[0.3mm]

\begin{oz}
Var = ...
\end{oz}
&variable assignment (only single-assignment)\\
\hline

\begin{oz}
declare Var 
\end{oz}
&global declaration of Var\\
\hline
 
\begin{oz}
local Var in 
  ...
end
\end{oz}
&local declaration\\
\hline
 
 
\begin{oz}
fun {FunName Arg1 ... ArgN}
  ...
end
\end{oz}
&function definition\\
\hline
 
\begin{oz}
proc {ProcName Arg1 ... ArgN}
  ...
end
\end{oz}
&procedure definition\\
\hline

\begin{oz}
if Condition1 then ...
elseif Condition2 then ...
else ...
end
\end{oz}
&if \dots else if \dots else \dots \\
\hline
 
 
\begin{oz}
case Var of Pattern_1 then ...
[] Pattern_2 then ...
else ...
end
\end{oz}
&pattern matching \\[0.4em]



%%%%%
\midrule[0.3mm]
\multicolumn{2}{c}{\textbf{Boolean expressions and operators}}\\
\midrule[0.3mm]



\begin{oz}
false 
\end{oz}
&false value\\
\hline
 
\begin{oz}
true
\end{oz}
&true value\\
\hline
 
\begin{oz}
andthen
\end{oz}
&logical AND \\
\hline
 
\begin{oz}
orelse
\end{oz}
&logical OR\\
\hline
 
\begin{oz}
==
\end{oz}
&logical equality\\
\hline
 
\begin{oz}
\=
\end{oz}
&logical inequality\\
\hline
 
\begin{oz}
{Not [Your Expression]}
\end{oz}
&logical NOT\\[0.4em]


%%%%% 
\midrule[0.3mm]
\multicolumn{2}{c}{\textbf{Comparison operators}}\\
\midrule[0.3mm]



\begin{oz}
<
\end{oz}
&less than\\
\hline
 
\begin{oz}
=<
\end{oz}
&less than or equal to (because $< $= is an arrow)\\
\hline
 
\begin{oz}
>
\end{oz}
&greater than\\
\hline
 
\begin{oz}
>=
\end{oz}
&greater than or equal to\\[0.4em]


%%%%% 
\midrule[0.3mm]
\multicolumn{2}{c}{\textbf{Arithmetic operators}}\\
\midrule[0.3mm]



\begin{oz}
+
\end{oz}
&addition\\
\hline
 
\begin{oz}
-
\end{oz}
&subtraction\\
\hline
 
\begin{oz}
*
\end{oz}
&multiplication\\
\hline
 
\begin{oz}
/
\end{oz}
&division (for floating point numbers)\\
\hline
 
\begin{oz}
div
\end{oz}
&division (for integers)\\
\hline
 
\begin{oz}
mod
\end{oz}
&modulo\\
\hline
 
\begin{oz}
{Pow A B}
\end{oz}
&$A^{B}$\\
\hline
 
\begin{oz}
{Abs A}
\end{oz}
&absolute value of A\\
\hline

\begin{oz}
E = ~1
\end{oz}
&unary negation (because - is an operator) \\[0.4em]

 
 
 %%%%% 
\midrule[0.3mm]
\multicolumn{2}{c}{\textbf{Data structures}}\\
\midrule[0.3mm]



\begin{oz}
S = "A string"
\end{oz}
& string declaration\\
\hline

\begin{oz}
A = hELLO
\end{oz}
& atom declaration (with a lowercase letter first)\\
\hline

\begin{oz}
A = 'An atom'
\end{oz}
& same (with an uppercase letter first, and a space)\\
\hline



\begin{oz}
X = label(feature1:Field1 
		... 
	  featureN:FieldN)
\end{oz}
&record structure (features and label are atoms) \\
\hline

\begin{oz}
R.feature
\end{oz}
&access to the record's fields\\
\hline

\begin{oz}
T = 1#2#3
\end{oz}
& common operator (T = '\#'(1:1 2:2 3:3))\\
\hline

\begin{oz}
L = '|'(1:1 2:'|'(1:2 2:nil))
\end{oz}
& list structure\\
\hline

\begin{oz}
L = 1|2|nil
\end{oz}
& a syntactic sugar to declare a list\\
\hline

\begin{oz}
L = [1 2]
\end{oz}
& another syntactic sugar for list declaration\\[0.4em]



 %%%%% 
\midrule[0.3mm]
\multicolumn{2}{c}{\textbf{Explicit state}}\\
\midrule[0.3mm]



\begin{oz}
X = {NewCell Y}
\end{oz}
&cell creation (multiple assignment variable) \\
\hline

\begin{oz}
@X
\end{oz}
& access to the cell's current content\\
\hline

\begin{oz}
X := Z
\end{oz}
&changes the content of the cell \\
\hline

\begin{oz}
for X in L do 
  ...
end
\end{oz}
&foreach loop (used with lists)\\
\hline

\begin{oz}
for X in 1..N do 
  ...
end
\end{oz}
&traditional for loop \\[0.4em]



\midrule[0.3mm]
 %%%%% 
\multicolumn{2}{c}{\textbf{Object-oriented programming}}\\
\midrule[0.3mm]



\begin{oz}
class AClass
	attr a1 ... an
	meth init(Arg) ... end
	meth m1 ... end
		...
	meth mn(Arg) ... end
end
\end{oz}
&class definition\\
\hline

\begin{oz}
X = {New AClass init('arg')}
{X m1}
\end{oz}
&object creation and use\\[0.4em]



\midrule[0.3mm]
 %%%%% 
\multicolumn{2}{c}{\textbf{Exception handling}}\\
\midrule[0.3mm]



\begin{oz}
raise E end
\end{oz}
&throws an exception E \\
\hline

\begin{oz}
try ... catch X then ... end
\end{oz}
&catches a raised exception\\[0.4em]



\midrule[0.3mm]
%%%%% 
\multicolumn{2}{c}{\textbf{Concurrent programming}}\\
\midrule[0.3mm]



\begin{oz}
thread ... end
\end{oz}
&thread creation\\


\end{longtable}



\end{document}
